% Options for packages loaded elsewhere
\PassOptionsToPackage{unicode}{hyperref}
\PassOptionsToPackage{hyphens}{url}
%
\documentclass[
]{report}
\usepackage{lmodern}
\usepackage{amssymb,amsmath}
\usepackage{ifxetex,ifluatex}
\ifnum 0\ifxetex 1\fi\ifluatex 1\fi=0 % if pdftex
  \usepackage[T1]{fontenc}
  \usepackage[utf8]{inputenc}
  \usepackage{textcomp} % provide euro and other symbols
\else % if luatex or xetex
  \usepackage{unicode-math}
  \defaultfontfeatures{Scale=MatchLowercase}
  \defaultfontfeatures[\rmfamily]{Ligatures=TeX,Scale=1}
\fi
% Use upquote if available, for straight quotes in verbatim environments
\IfFileExists{upquote.sty}{\usepackage{upquote}}{}
\IfFileExists{microtype.sty}{% use microtype if available
  \usepackage[]{microtype}
  \UseMicrotypeSet[protrusion]{basicmath} % disable protrusion for tt fonts
}{}
\makeatletter
\@ifundefined{KOMAClassName}{% if non-KOMA class
  \IfFileExists{parskip.sty}{%
    \usepackage{parskip}
  }{% else
    \setlength{\parindent}{0pt}
    \setlength{\parskip}{6pt plus 2pt minus 1pt}}
}{% if KOMA class
  \KOMAoptions{parskip=half}}
\makeatother
\usepackage{xcolor}
\IfFileExists{xurl.sty}{\usepackage{xurl}}{} % add URL line breaks if available
\IfFileExists{bookmark.sty}{\usepackage{bookmark}}{\usepackage{hyperref}}
\hypersetup{
  pdftitle={1 Report ASP20 Boost},
  pdfauthor={Johannes Strauß; Levin Wiebelt; Sebastian Aristizabal},
  hidelinks,
  pdfcreator={LaTeX via pandoc}}
\urlstyle{same} % disable monospaced font for URLs
\usepackage[left=4cm, right=3cm, top=2.5cm, bottom=2.5cm]{geometry}
\usepackage{color}
\usepackage{fancyvrb}
\newcommand{\VerbBar}{|}
\newcommand{\VERB}{\Verb[commandchars=\\\{\}]}
\DefineVerbatimEnvironment{Highlighting}{Verbatim}{commandchars=\\\{\}}
% Add ',fontsize=\small' for more characters per line
\usepackage{framed}
\definecolor{shadecolor}{RGB}{248,248,248}
\newenvironment{Shaded}{\begin{snugshade}}{\end{snugshade}}
\newcommand{\AlertTok}[1]{\textcolor[rgb]{0.94,0.16,0.16}{#1}}
\newcommand{\AnnotationTok}[1]{\textcolor[rgb]{0.56,0.35,0.01}{\textbf{\textit{#1}}}}
\newcommand{\AttributeTok}[1]{\textcolor[rgb]{0.77,0.63,0.00}{#1}}
\newcommand{\BaseNTok}[1]{\textcolor[rgb]{0.00,0.00,0.81}{#1}}
\newcommand{\BuiltInTok}[1]{#1}
\newcommand{\CharTok}[1]{\textcolor[rgb]{0.31,0.60,0.02}{#1}}
\newcommand{\CommentTok}[1]{\textcolor[rgb]{0.56,0.35,0.01}{\textit{#1}}}
\newcommand{\CommentVarTok}[1]{\textcolor[rgb]{0.56,0.35,0.01}{\textbf{\textit{#1}}}}
\newcommand{\ConstantTok}[1]{\textcolor[rgb]{0.00,0.00,0.00}{#1}}
\newcommand{\ControlFlowTok}[1]{\textcolor[rgb]{0.13,0.29,0.53}{\textbf{#1}}}
\newcommand{\DataTypeTok}[1]{\textcolor[rgb]{0.13,0.29,0.53}{#1}}
\newcommand{\DecValTok}[1]{\textcolor[rgb]{0.00,0.00,0.81}{#1}}
\newcommand{\DocumentationTok}[1]{\textcolor[rgb]{0.56,0.35,0.01}{\textbf{\textit{#1}}}}
\newcommand{\ErrorTok}[1]{\textcolor[rgb]{0.64,0.00,0.00}{\textbf{#1}}}
\newcommand{\ExtensionTok}[1]{#1}
\newcommand{\FloatTok}[1]{\textcolor[rgb]{0.00,0.00,0.81}{#1}}
\newcommand{\FunctionTok}[1]{\textcolor[rgb]{0.00,0.00,0.00}{#1}}
\newcommand{\ImportTok}[1]{#1}
\newcommand{\InformationTok}[1]{\textcolor[rgb]{0.56,0.35,0.01}{\textbf{\textit{#1}}}}
\newcommand{\KeywordTok}[1]{\textcolor[rgb]{0.13,0.29,0.53}{\textbf{#1}}}
\newcommand{\NormalTok}[1]{#1}
\newcommand{\OperatorTok}[1]{\textcolor[rgb]{0.81,0.36,0.00}{\textbf{#1}}}
\newcommand{\OtherTok}[1]{\textcolor[rgb]{0.56,0.35,0.01}{#1}}
\newcommand{\PreprocessorTok}[1]{\textcolor[rgb]{0.56,0.35,0.01}{\textit{#1}}}
\newcommand{\RegionMarkerTok}[1]{#1}
\newcommand{\SpecialCharTok}[1]{\textcolor[rgb]{0.00,0.00,0.00}{#1}}
\newcommand{\SpecialStringTok}[1]{\textcolor[rgb]{0.31,0.60,0.02}{#1}}
\newcommand{\StringTok}[1]{\textcolor[rgb]{0.31,0.60,0.02}{#1}}
\newcommand{\VariableTok}[1]{\textcolor[rgb]{0.00,0.00,0.00}{#1}}
\newcommand{\VerbatimStringTok}[1]{\textcolor[rgb]{0.31,0.60,0.02}{#1}}
\newcommand{\WarningTok}[1]{\textcolor[rgb]{0.56,0.35,0.01}{\textbf{\textit{#1}}}}
\usepackage{graphicx,grffile}
\makeatletter
\def\maxwidth{\ifdim\Gin@nat@width>\linewidth\linewidth\else\Gin@nat@width\fi}
\def\maxheight{\ifdim\Gin@nat@height>\textheight\textheight\else\Gin@nat@height\fi}
\makeatother
% Scale images if necessary, so that they will not overflow the page
% margins by default, and it is still possible to overwrite the defaults
% using explicit options in \includegraphics[width, height, ...]{}
\setkeys{Gin}{width=\maxwidth,height=\maxheight,keepaspectratio}
% Set default figure placement to htbp
\makeatletter
\def\fps@figure{htbp}
\makeatother
\setlength{\emergencystretch}{3em} % prevent overfull lines
\providecommand{\tightlist}{%
  \setlength{\itemsep}{0pt}\setlength{\parskip}{0pt}}
\setcounter{secnumdepth}{-\maxdimen} % remove section numbering

\title{1 Report ASP20 Boost}
\usepackage{etoolbox}
\makeatletter
\providecommand{\subtitle}[1]{% add subtitle to \maketitle
  \apptocmd{\@title}{\par {\large #1 \par}}{}{}
}
\makeatother
\subtitle{First Report}
\author{Johannes Strauß \and Levin Wiebelt \and Sebastian Aristizabal}
\date{27 Mai 2020}

\begin{document}
\maketitle
\begin{abstract}
This is the abstract.

It consists of two paragraphs.
\end{abstract}

{
\setcounter{tocdepth}{1}
\tableofcontents
}
\hypertarget{how-to-cite-and-configure-the-documents-appearance}{%
\chapter{How to cite and configure the document's
appearance}\label{how-to-cite-and-configure-the-documents-appearance}}

If you don't have Latex download latex for R with
\texttt{tinytex::install\_tinytex()}.

\hypertarget{write}{%
\section{Write}\label{write}}

To cite:

\begin{itemize}
\tightlist
\item
  Parentheses: Implementation builds upon the asp20 model (Riebl 2020).
\item
  Inline: It is implemented as R6 class Chang (2019).
\item
  Including pages: For the theoretical foundations of our
  implementation, we rely mostly on (Fahrmeir 2013, 219) and (Hastie,
  Tibshirani, and Friedman 2009, 358).
\end{itemize}

In the \texttt{report} folder there's two \texttt{.bib} files. These
contain all reference we should need. To update the \texttt{package.bib}
write the name of the package in the following chunk and \emph{actively}
run it (It should be installed in your terminal). The\texttt{citavi.bib}
file needs to be exported from citavi.

\begin{Shaded}
\begin{Highlighting}[]
\CommentTok{# automatically create a bib database for R packages}
\NormalTok{knitr}\OperatorTok{::}\KeywordTok{write_bib}\NormalTok{(}\KeywordTok{c}\NormalTok{(}
  \KeywordTok{.packages}\NormalTok{(), }\StringTok{'R6'}\NormalTok{, }\StringTok{'knitr'}\NormalTok{, }\StringTok{'rmarkdown'}\NormalTok{, }\StringTok{'asp20model'}\NormalTok{, }\StringTok{'gamboostLSS'}\NormalTok{,}\StringTok{'mboost'}\NormalTok{, }\StringTok{'tidyverse'}  
\NormalTok{), }\StringTok{'packages.bib'}\NormalTok{)}
\end{Highlighting}
\end{Shaded}

To cite write a `\texttt{@} followed by \texttt{firstname.year} as shown
in the \texttt{.bib} files for reference. See further
\href{https://rmarkdown.rstudio.com/authoring_bibliographies_and_citations.html}{cite}

As a tip hit always enter after each period for better readability in R
Studio. Like this.

Two enters give me a new paragraph.

\hypertarget{including-plots}{%
\section{Including Plots}\label{including-plots}}

You can also embed plots, for example:

\includegraphics{1_REPORT_files/figure-latex/pressure-1.pdf}

\hypertarget{appearance-using-the-yaml-header}{%
\section{\texorpdfstring{Appearance using the \texttt{YAML}
Header:}{Appearance using the YAML Header:}}\label{appearance-using-the-yaml-header}}

We can regulate the appearance of the document from here. There's a lot
flexibility but the learning curve is somehow steep. I'll link useful
resources to learn, but, in principle, the document is \emph{ready for
writing}.

Here the corresponding references:

\begin{itemize}
\tightlist
\item
  \href{https://bookdown.org/yihui/rmarkdown/pdf-document.html}{PDF
  output}
\item
  \href{https://pandoc.org/MANUAL.html\#templates}{Pandoc Manuals}
\item
  \href{https://bookdown.org/yihui/rmarkdown/pdf-document.html\#table-of-contents-1}{TOC}
\item
  \href{https://github.com/jgm/pandoc-citeproc/blob/master/man/pandoc-citeproc.1.md}{Cite
  with Pandoc}
\end{itemize}

Here those inputs that are easy to configuration for you guys to get an
idea:

\begin{itemize}
\tightlist
\item
  With \texttt{documentclass} has three base templates:
  \texttt{article}, \texttt{report} and \texttt{book}.
\item
  \texttt{geometry} let's me change the margins.
\item
  \texttt{lof} and \texttt{lot} set to true give me a list of figures
  and tables respectively.
\end{itemize}

\hypertarget{introduction}{%
\chapter{1. Introduction:}\label{introduction}}

\begin{itemize}
\tightlist
\item
  The developement of this package takes place in the framework of the
  seminar ``Advanced Statisical Programming'' -\textgreater{} SS20
\item
  We expand the R6 class \emph{``LocationScaleRegression''} belonging to
  the \texttt{asp20model} package concieved specifficaly for this
  seminar (Riebl 2020)
\item
  Our aim is to develope an multi-faceted implementation of boosting for
  location and scale including component-wise boosting, use of
  cross-validation to determine the optimal stopping critera and useful
  options for visualization.
\end{itemize}

\hypertarget{description-of-progress}{%
\chapter{2. Description of progress}\label{description-of-progress}}

\begin{quote}
Hannes lead question: What are the critical aspects of your
implementation and your package? Stability? Performance?
User-friendliness? How are you going to deal with these potential
issues?
\end{quote}

\begin{quote}
\begin{enumerate}
\def\labelenumi{\arabic{enumi}.}
\setcounter{enumi}{1}
\tightlist
\item
  What design designs did you make with respect to your implementation?
  How are you going to extend the R6 class? What additional functions
  are you going to provide?
\end{enumerate}
\end{quote}

I propose a timeline where the implementation process is described step
by step as a list e.g:

\begin{enumerate}
\def\labelenumi{\arabic{enumi}.}
\tightlist
\item
  Primitive implementation of boosting
\item
  Simple boosting for location implemented
\item
  Adressed the estimation for the variance of gamma.
\item
  Johannes finished everything \#lol. 27.05.
\end{enumerate}

\hypertarget{johannes-is-gzus.}{%
\section{\texorpdfstring{Johannes is
\emph{gzus}.}{Johannes is gzus.}}\label{johannes-is-gzus.}}

\hypertarget{description-of-the-problem}{%
\chapter{3. Description of the
problem}\label{description-of-the-problem}}

\begin{quote}
Hannes lead question: Which open questions do you have about the
statistical model and methodology?
\end{quote}

\hypertarget{extensions-to-be-implemented.}{%
\chapter{4. Extensions to be
implemented.}\label{extensions-to-be-implemented.}}

\begin{quote}
Hannes' Lead question: What functionality are you planning to (or did
you already) implement? What did you decide to leave out?
\end{quote}

\hypertarget{summary}{%
\chapter{5. Summary?}\label{summary}}

\hypertarget{references}{%
\chapter*{6. References}\label{references}}
\addcontentsline{toc}{chapter}{6. References}

\hypertarget{refs}{}
\leavevmode\hypertarget{ref-R-R6}{}%
Chang, Winston. 2019. \emph{R6: Encapsulated Classes with Reference
Semantics}. \url{https://CRAN.R-project.org/package=R6}.

\leavevmode\hypertarget{ref-Fahrmeir.2013}{}%
Fahrmeir, Ludwig. 2013. \emph{Regression: Models, Methods and
Applications}. New York: Springer.

\leavevmode\hypertarget{ref-Hastie.2009}{}%
Hastie, Trevor, Robert Tibshirani, and J. H. Friedman. 2009. \emph{The
Elements of Statistical Learning: Data Mining, Inference, and Prediction
/ Trevor Hastie, Robert Tibshirani, Jerome Friedman}. 2nd ed. Springer
Series in Statistics. New York: Springer.

\leavevmode\hypertarget{ref-R-asp20model}{}%
Riebl, Hannes. 2020. \emph{Asp20model: An R6 Class for Location-Scale
Regression Models}.

\end{document}
