% Options for packages loaded elsewhere
\PassOptionsToPackage{unicode}{hyperref}
\PassOptionsToPackage{hyphens}{url}
%
\documentclass[
]{report}
\usepackage{lmodern}
\usepackage{amssymb,amsmath}
\usepackage{ifxetex,ifluatex}
\ifnum 0\ifxetex 1\fi\ifluatex 1\fi=0 % if pdftex
  \usepackage[T1]{fontenc}
  \usepackage[utf8]{inputenc}
  \usepackage{textcomp} % provide euro and other symbols
\else % if luatex or xetex
  \usepackage{unicode-math}
  \defaultfontfeatures{Scale=MatchLowercase}
  \defaultfontfeatures[\rmfamily]{Ligatures=TeX,Scale=1}
\fi
% Use upquote if available, for straight quotes in verbatim environments
\IfFileExists{upquote.sty}{\usepackage{upquote}}{}
\IfFileExists{microtype.sty}{% use microtype if available
  \usepackage[]{microtype}
  \UseMicrotypeSet[protrusion]{basicmath} % disable protrusion for tt fonts
}{}
\makeatletter
\@ifundefined{KOMAClassName}{% if non-KOMA class
  \IfFileExists{parskip.sty}{%
    \usepackage{parskip}
  }{% else
    \setlength{\parindent}{0pt}
    \setlength{\parskip}{6pt plus 2pt minus 1pt}}
}{% if KOMA class
  \KOMAoptions{parskip=half}}
\makeatother
\usepackage{xcolor}
\IfFileExists{xurl.sty}{\usepackage{xurl}}{} % add URL line breaks if available
\IfFileExists{bookmark.sty}{\usepackage{bookmark}}{\usepackage{hyperref}}
\hypersetup{
  pdftitle={Report ASP20 Boost},
  pdfauthor={Johannes Strauß; Levin Wiebelt; Sebastian Aristizabal},
  hidelinks,
  pdfcreator={LaTeX via pandoc}}
\urlstyle{same} % disable monospaced font for URLs
\usepackage[left=4cm, right=3cm, top=2.5cm, bottom=2.5cm]{geometry}
\setlength{\emergencystretch}{3em} % prevent overfull lines
\providecommand{\tightlist}{%
  \setlength{\itemsep}{0pt}\setlength{\parskip}{0pt}}
\setcounter{secnumdepth}{-\maxdimen} % remove section numbering

\title{Report ASP20 Boost}
\usepackage{etoolbox}
\makeatletter
\providecommand{\subtitle}[1]{% add subtitle to \maketitle
  \apptocmd{\@title}{\par {\large #1 \par}}{}{}
}
\makeatother
\subtitle{First Report}
\author{Johannes Strauß \and Levin Wiebelt \and Sebastian Aristizabal}
\date{04 Juni 2020}

\begin{document}
\maketitle

{
\setcounter{tocdepth}{1}
\tableofcontents
}
\hypertarget{introduction}{%
\chapter{1. Introduction:}\label{introduction}}

The course ``Advanced Statistical Programming with R'' consists in
implementing a package to address location-scale regression. The sample
dataset consists of a response y, that's expectation is in linear
relationship to a set of predictors xi. Its variance, in contrast, is
dependent on a different set of linear predictors zi, however
transformed by the response function g(x) = exp(x). This ensures
positiveness of the variances. The goal approach is to estimate the
effects of both: xi on the expectation of y (location) - this is
captured by the estimates beta, and zi on the variance of y (scale) -
this is captured by the estimates gamma. Our student group
\texttt{asp20boost} solves this task applying the concept of boosting.

In part 2 of this report we explain the concept of boosting and
elaborate on how it suites to address the problem of location-scale
regression. In the following part 3 we describe our implementation
milestones, whereby part 4 especially elaborates on componentwise
boosting. In part 5 we present our thoughts on upcoming challenges and
critical aspects of the package development.

\hypertarget{concept-boosting}{%
\chapter{2. Concept: Boosting}\label{concept-boosting}}

The first step in our progress was to understand the theoretical concept
of boosting, and which types of problems it is able to solve. In the
following the term ``booster'' refers to a boosting algorithm.

\hypertarget{location-booster}{%
\section{Location-Booster}\label{location-booster}}

To explain the concept and follow our process of understanding we first
explain boosting for location parameters. Boosting scale parameters
requires another way of thinking about it and constituted a first
milestone in our understanding and implementation. The boosting concept
relies on the idea of iterative estimation. To estimate the location
parameters beta via boosting, one starts with initial estimates, which
may be far from optimal. However, with estimates and the response vector
at hand, residuals can be calculated. The boosting-algortihm focuses on
those calculated residuals. The effect of the predictors xi on these
residuals is estimated by least squares and the resulting effect is
added to the location-estimate beta, adjusted by some learning rate.
This yields a better fit of the location-model, which results in smaller
residuals. These new residuals are estimated again in the next
iteration, yielding smaller effect sizes, and hence a convergence of the
boosting algorithm towards the OLS-estimate. A simple booster stops
after a fixed number of iterations is reached. \textbf{(comp. Kneib et
al)}

\hypertarget{scale-booster}{%
\section{Scale-Booster}\label{scale-booster}}

{[}Loss-function erklären: Boosting can be viewed as lowering
iteratively the amount of a loss function, which is in our case the
score of the normal distribution{]}

The concept for boosting scale parameters is the same. However,
residuals for the variance estimates can not be calculated, since there
is no such thing as ``observed'' variances for individual response
values. The solution is to substitute the calculated residuals with a
different term: the derivative of the loss function, which in our case
is the log-likelihood of the normal distribution. Instead of estiamting
stepwise residuals, one estiamtes hence stepwise the score.

The resulting loss fcuntion then, is defined by sum of squares oa the
term: ui (=deviance-residuals) - inverse predicted scores *
working-residuals

We employed the resid-function of the asp20model class, on which our
class is built on.

\hypertarget{implementation}{%
\chapter{3. Implementation}\label{implementation}}

In the current version of the asp20boost-package, the boosting algorithm
is already implemented in order to allow componentwise boosting, which
will be explained in part 4. For good explanation and for following our
development process in this part we first describe the implementation of
ordinary boosting - which however does not appear in our code anymore.

A simple booster for the location parameters beta works without
extending the LocationScaleRegression-Class. The current residuals may
be extrated with the resid()-method, then estimated. The location
parameter is updated. This causes the LocationScaleRegression-Class to
calculate updated state-dependent objects, such as the loglikelihood,
gradients and the residuals. This is what's needed to repeat the
procedure of estimating residuals and updating location parameters.

The simple booster for gamma consists in the same two repeating steps of
first estimating a term, and second updating the scale parameters gamma
by adding the estimated effect - adjusted by a learning rate. The term
to be estimated, however, is not the residual, but the derivative of the
loglikelihood of the normal distribution. To calculate this, our code
makes use of the resid-function(), but this time passing the argument
``deviance'', which results in residuals adjusted by the fitted scale
estimates.

{[}Is the gamma booster dependent on simultaneous running of beta
booster?{]}

Having implemented a boosting algorithm for location and scale, the
code-basis for our package was achieved.

\hypertarget{functionality-componentwise-boosting}{%
\chapter{4. Functionality: Componentwise
Boosting}\label{functionality-componentwise-boosting}}

A usefull functionality of boosting is componentwise boosting. The idea
is to not update a whole parameter vector, but only one entry of it. The
entry chosen for the update at this juncture is the one yielding the
best improvement in the sense of lowering the loss function. The loss
function in our case is the loglikelihood of the normally distributed
data. Hence in each iteration only one component of the location-, as
well as one component of the scale-parameters are estimated.

The implementation in our package works via the extension of the
LocationScaleRegressionClass by two active fields
``bestFittingVariableBeta'' and ``bestFittingVariableGamma''. These
functions partition the design matrix X, respectively Z, into its single
columns and then estimate the residuals - respectively the score -
seperately for each component, and determine loss functions for a
hypothetical update with the respective component. Comparing with the
old loss function value, the highest loss-improvement may be determined
and the respective component is used to update the parameter-vector.

We are in the process of reconsidering the design of this
implementation. Other design possibilities are the following:

\begin{itemize}
\tightlist
\item
  create public fields for the heavily used score-function values
\item
  move the calculation of componentwise losses to an external function
\item
  harmonize boosting and componentwise boosting into one external
  function, determining the mode of operation by an argument
  `componentwise = TRUE'
\end{itemize}

Another conceptual questions that comes up is if the best
loss-improvement may be indicated by the already existing gradients, and
hence there is no need for extra calculation.

\hypertarget{prospects}{%
\chapter{5. Prospects}\label{prospects}}

\hypertarget{further-functionalities}{%
\section{Further Functionalities}\label{further-functionalities}}

The most important functionality we intend to implement in our package
is variable selection via cross validation. Working out the theoretical
concept behind this idea is one of our next milestones.

One further, rather loose, idea is to find ways to optimize the learning
rate in our boosting algorithm.

\hypertarget{stability}{%
\section{Stability}\label{stability}}

We intend to implement further automated unit tests. This will enable us
to assess quickly the stability of our code given differing inputs.
Departing from this point we intend to reduce proneness to input errors.

\hypertarget{performance}{%
\section{Performance}\label{performance}}

\begin{itemize}
\tightlist
\item
  A major performance problem in our code is that small learning rates
  for the beta-booster cause processing time of the code to increase
  sharply. This remains to be solved.
\end{itemize}

*A good choice of starting values for beta and gamma may enhance
performance. A possible candidate is the mean of the response.

\hypertarget{further}{%
\section{Further}\label{further}}

\begin{itemize}
\item
  Allow user-input of the LocationScaleRegressionBoost-model in form of
  a dataframe, for example by allowing to pass an optional
  ``data''-argument as known from the lm-call. This may reduce proneness
  to input-errors.
\item
  Visualize the results of the boosted estimates
\item
  Document the extension of the R6-Class properly and inherit
  documentation of the LocationScaleRegression-Class.
\end{itemize}

\hypertarget{application}{%
\section{Application}\label{application}}

{[}Johannes: Munich Rent Data? - Identify Outliers{]}

\end{document}
